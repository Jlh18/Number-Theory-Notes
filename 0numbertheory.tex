\documentclass{book}

\usepackage[left=1in,right=1in]{geometry}
\usepackage{subfiles}
\usepackage{amsmath, amssymb, stmaryrd, verbatim} % math symbols
\usepackage{amsthm} % thm environment
\usepackage{mdframed} % Customizable Boxes
\usepackage{hyperref,nameref,cleveref,enumitem} % for references, hyperlinks
\usepackage[dvipsnames]{xcolor} % Fancy Colours
\usepackage{mathrsfs} % Fancy font
\usepackage{tikz, tikz-cd, float} % Commutative Diagrams
\usepackage{perpage}
\usepackage{parskip} % So that paragraphs look nice
\usepackage{ifthen,xargs} % For defining better commands
\usepackage{anyfontsize}
\usepackage[T1]{fontenc}
\usepackage[utf8]{inputenc}
\usepackage{tgpagella}
\usepackage{mathtools}
\usepackage{cancel}

% Shortcuts

% % Misc
\newcommand{\brkt}[1]{\left(#1\right)}
\newcommand{\sqbrkt}[1]{\left[#1\right]}
\newcommand{\dash}{\text{-}}
\newcommand{\ontop}[2]{\stackrel{\mathclap{#1}}{#2}}

% % Logic
\renewcommand{\implies}{\Rightarrow}
\renewcommand{\iff}{\Leftrightarrow}
\newcommand{\limplies}{\Leftarrow}
\newcommand{\NOT}{\neg\,}
\newcommand{\AND}{\, \land \,}
\newcommand{\OR}{\, \lor \,}
\newenvironment{forward}{($\implies$)}{}
\newenvironment{backward}{($\limplies$)}{}

% % Sets
\DeclareMathOperator{\supp}{supp}
\newcommand{\set}[1]{\left\{#1\right\}}
\newcommand{\st}{\,|\,}
\newcommand{\minus}{\setminus}
\newcommand{\subs}{\subseteq}
\newcommand{\ssubs}{\subsetneq}
\DeclareMathOperator{\im}{Im}
\newcommand{\nothing}{\varnothing}
\newcommand\res[2]{{% we make the whole thing an ordinary symbol
  \left.\kern-\nulldelimiterspace 
  % automatically resize the bar with \right
  #1 % the function
  \vphantom{\big|} 
  % pretend it's a little taller at normal size
  \right|_{#2} % this is the delimiter
  }}

% % Greek 
\newcommand{\al}{\alpha}
\newcommand{\be}{\beta}
\newcommand{\ga}{\gamma}
\newcommand{\de}{\delta}
\newcommand{\ep}{\varepsilon}
\newcommand{\io}{\iota}
\newcommand{\ka}{\kappa}
\newcommand{\la}{\lambda}
\newcommand{\om}{\omega}
\newcommand{\si}{\sigma}

\newcommand{\De}{\Delta}
\newcommand{\Si}{\Sigma}
\newcommand{\Om}{\Omega}

% % Mathbb
\newcommand{\N}{\mathbb{N}}
\newcommand{\M}{\mathbb{M}}
\newcommand{\Z}{\mathbb{Z}}
\newcommand{\Q}{\mathbb{Q}}
\newcommand{\R}{\mathbb{R}}
\newcommand{\C}{\mathbb{C}}
\newcommand{\F}{\mathbb{F}}
\newcommand{\bP}{\mathbb{P}}
\newcommand{\V}{\mathbb{V}}
\newcommand{\U}{\mathbb{U}}

% % Mathcal
\renewcommand{\AA}{\mathcal{A}}
\newcommand{\BB}{\mathcal{B}}
\newcommand{\CC}{\mathcal{C}}
\newcommand{\DD}{\mathcal{D}}
\newcommand{\EE}{\mathcal{E}}
\newcommand{\FF}{\mathcal{F}}
\newcommand{\GG}{\mathcal{G}}
\newcommand{\HH}{\mathcal{H}}
\newcommand{\II}{\mathcal{I}}
\newcommand{\JJ}{\mathcal{J}}
\newcommand{\KK}{\mathcal{K}}
\newcommand{\LL}{\mathcal{L}}
\newcommand{\MM}{\mathcal{M}}
\newcommand{\NN}{\mathcal{N}}
\newcommand{\OO}{\mathcal{O}}
\newcommand{\PP}{\mathcal{P}}
\newcommand{\QQ}{\mathcal{Q}}
\newcommand{\RR}{\mathcal{R}}
\renewcommand{\SS}{\mathcal{S}}
\newcommand{\TT}{\mathcal{T}}
\newcommand{\UU}{\mathcal{U}}
\newcommand{\VV}{\mathcal{V}}
\newcommand{\WW}{\mathcal{W}}
\newcommand{\XX}{\mathcal{X}}
\newcommand{\YY}{\mathcal{Y}}
\newcommand{\ZZ}{\mathcal{Z}}

% % Mathfrak
\newcommand{\f}[1]{\mathfrak{#1}}

% % Mathrsfs
\newcommand{\s}[1]{\mathscr{#1}}

% % Category Theory
\newcommand{\obj}[1]{\mathrm{Obj}\left(#1\right)}
\newcommand{\Hom}[3]{\mathrm{Hom}_{#3}(#1, #2)\,}
\newcommand{\mor}[3]{\mathrm{Mor}_{#3}(#1, #2)\,}
\newcommand{\End}[2]{\mathrm{End}_{#2}#1\,}
\newcommand{\aut}[2]{\mathrm{Aut}_{#2}#1\,}
\newcommand{\CAT}{\mathbf{Cat}}
\newcommand{\SET}{\mathbf{Set}}
\newcommand{\TOP}{\mathbf{Top}}
\newcommand{\GRP}{\mathbf{Grp}}
\newcommand{\RING}{\mathbf{Ring}}
\newcommand{\MOD}[1][R]{#1\text{-}\mathbf{Mod}}
\newcommand{\VEC}[1][K]{#1\text{-}\mathbf{Vec}}
\newcommand{\ALG}[1][R]{#1\text{-}\mathbf{Alg}}
\newcommand{\PSH}[1]{\mathbf{PSh}\brkt{#1}}
\newcommand{\map}[4]{#1 \yrightarrow[#4][2.5pt]{#3}[-1pt] #2}
\newcommand{\op}{^{op}}
\newcommand{\darrow}{\downarrow}
\newcommand{\LIM}[2]{\varprojlim_{#2}#1}
\newcommand{\COLIM}[2]{\varinjlim_{#2}#1}

% % Algebra
\newcommand{\iso}{\cong}
\newcommand{\nsub}{\trianglelefteq}
\newcommand{\id}[1]{\mathrm{id}_{#1}}
\newcommand{\inv}{^{-1}}

% % Analysis
\newcommand{\abs}[1]{\left\vert #1 \right\vert}
\newcommand{\norm}[1]{\left\Vert #1 \right\Vert}
\renewcommand{\bar}[1]{\overline{#1}}
\newcommand{\<}{\langle}
\renewcommand{\>}{\rangle}
\renewcommand{\hat}[1]{\widehat{#1}}
\renewcommand{\check}[1]{\widecheck{#1}}
\newcommand{\dbd}[2]{\frac{\partial #1}{\partial #2}}

% % Galois
\newcommand{\Gal}[2]{\mathrm{Gal}_{#1}(#2)}
\DeclareMathOperator{\Orb}{Orb}
\DeclareMathOperator{\Stab}{Stab}
\newcommand{\emb}[3]{\mathrm{Emb}_{#1}(#2, #3)}
\newcommand{\Char}[1]{\mathrm{Char}(#1)}

% % Number Theory
\newcommand{\legen}[2]{\brkt{\frac{#1}{#2}}}

% % Model Theory
\newcommand{\intp}[2]{
    \star_{\text{\scalebox{0.7}{$#1$}}}^{
    \text{\scalebox{0.7}{$#2$}}}}
\newcommand{\subintp}[3]{
    {#3}_{\text{\scalebox{0.7}{$#1$}}}^{
    \text{\scalebox{0.7}{$#2$}}}}
\newcommand{\modintp}[2]{#2^\text{\scalebox{0.7}{$#1$}}}
\newcommand{\mmintp}[1]{\modintp{\MM}{#1}}
\newcommand{\nnintp}[1]{\modintp{\NN}{#1}}
\newcommand{\const}[1]{{#1}_\mathrm{con}}
\newcommand{\func}[1]{{#1}_\mathrm{fun}}
\newcommand{\rel}[1]{{#1}_\mathrm{rel}}
\newcommand{\term}[1]{{#1}_\mathrm{ter}}
\newcommand{\struc}[1]{{#1}_\mathrm{str}}
\newcommand{\form}[1]{{#1}_\mathrm{for}}
\newcommand{\var}[1]{{#1}_\mathrm{var}}
\newcommand{\theory}[1]{{#1}_\mathrm{the}}
\newcommand{\carrier}[1]{{#1}_\mathrm{car}}
\newcommand{\model}[1]{\vDash_{#1}}
\newcommand{\nodel}[1]{\nvDash_{#1}}
\newcommand{\modelsi}{\model{\Si}}
\newcommand{\eldiag}[2]{\mathrm{ElDiag}(#1,#2)}
\newcommand{\atdiag}[2]{\mathrm{AtDiag}(#1,#2)}
\newcommand{\Theory}[1]{\mathrm{Th(#1)}}
\newcommand{\unisen}[1]{{#1}_\mathrm{uni}}
\newcommand{\lift}[2]{\uparrow_{#1}^{#2}}
\newcommand{\fall}[2]{\downarrow_{#1}^{#2}}
\newcommand{\Mod}[1]{\M \mathbf{od}(#1)}

%% code from mathabx.sty and mathabx.dcl to get some symbols from mathabx
\DeclareFontFamily{U}{mathx}{\hyphenchar\font45}
\DeclareFontShape{U}{mathx}{m}{n}{
      <5> <6> <7> <8> <9> <10>
      <10.95> <12> <14.4> <17.28> <20.74> <24.88>
      mathx10
      }{}
\DeclareSymbolFont{mathx}{U}{mathx}{m}{n}
\DeclareFontSubstitution{U}{mathx}{m}{n}
\DeclareMathAccent{\widecheck}{0}{mathx}{"71}

% Arrows with text above and below with adjustable displacement
% (Stolen from Stackexchange)
\newcommandx{\yaHelper}[2][1=\empty]{
\ifthenelse{\equal{#1}{\empty}}
  % no offset
  { \ensuremath{ \scriptstyle{ #2 } } } 
  % with offset
  { \raisebox{ #1 }[0pt][0pt]{ \ensuremath{ \scriptstyle{ #2 } } } }  
}

\newcommandx{\yrightarrow}[4][1=\empty, 2=\empty, 4=\empty, usedefault=@]{
  \ifthenelse{\equal{#2}{\empty}}
  % there's no text below
  { \xrightarrow{ \protect{ \yaHelper[ #4 ]{ #3 } } } } 
  % there's text below
  {
    \xrightarrow[ \protect{ \yaHelper[ #2 ]{ #1 } } ]
    { \protect{ \yaHelper[ #4 ]{ #3 } } } 
  } 
}

% xcolor
\definecolor{darkgrey}{gray}{0.10}
\definecolor{lightgrey}{gray}{0.30}
\definecolor{slightgrey}{gray}{0.80}
\definecolor{softblue}{RGB}{30,100,200}

% hyperref
\hypersetup{
      colorlinks = true,
      linkcolor = {softblue},
      citecolor = {blue}
}

\newcommand{\link}[1]{\hypertarget{#1}{}}
\newcommand{\linkto}[2]{\hyperlink{#1}{#2}}

% Perpage
\MakePerPage{footnote}

% Theorems

% % custom theoremstyles
\newtheoremstyle{definitionstyle}
{5pt}% above thm
{0pt}% below thm
{}% body font
{}% space to indent
{\bf}% head font
{\vspace{1mm}}% punctuation between head and body
{\newline}% space after head
{\thmname{#1}\thmnote{\,\,--\,\,#3}}

% % custom theoremstyles
\newtheoremstyle{propositionstyle}
{5pt}% above thm
{0pt}% below thm
{}% body font
{}% space to indent
{\bf}% head font
{\vspace{1mm}}% punctuation between head and body
{\newline}% space after head
{\thmname{#1}\thmnote{\,\,--\,\,#3}}

\newtheoremstyle{exercisestyle}%
{5pt}% above thm
{0pt}% below thm
{\it}% body font
{}% space to indent
{\scshape}% head font
{.}% punctuation between head and body
{ }% space after head
{\thmname{#1}\thmnote{ (#3)}}

\newtheoremstyle{remarkstyle}%
{5pt}% above thm
{0pt}% below thm
{}% body font
{}% space to indent
{\it}% head font
{.}% punctuation between head and body
{ }% space after head
{\thmname{#1}\thmnote{\,\,--\,\,#3}}

% % Theorem environments

\theoremstyle{definitionstyle}
\newmdtheoremenv[
    linewidth = 2pt,
    leftmargin = 20pt,
    rightmargin = 20pt,
    linecolor = darkgrey,
    topline = false,
    bottomline = false,
    rightline = false,
    footnoteinside = false
]{dfn}{Definition}
\newmdtheoremenv[
    linewidth = 2 pt,
    leftmargin = 20pt,
    rightmargin = 20pt,
    linecolor = darkgrey,
    topline = false,
    bottomline = false,
    rightline = false,
    footnoteinside = false
]{prop}{Proposition}
\newmdtheoremenv[
    linewidth = 2 pt,
    leftmargin = 20pt,
    rightmargin = 20pt,
    linecolor = darkgrey,
    topline = false,
    bottomline = false,
    rightline = false,
    footnoteinside = false
]{cor}{Corollary}


\theoremstyle{exercisestyle}
\newmdtheoremenv[
    linewidth = 0.7 pt,
    leftmargin = 20pt,
    rightmargin = 20pt,
    linecolor = darkgrey,
    topline = false,
    bottomline = false,
    rightline = false,
    footnoteinside = false
]{ex}{Exercise}
\newmdtheoremenv[
    linewidth = 0.7 pt,
    leftmargin = 20pt,
    rightmargin = 20pt,
    linecolor = darkgrey,
    topline = false,
    bottomline = false,
    rightline = false,
    footnoteinside = false
]{eg}{Example}
\newmdtheoremenv[
    linewidth = 0.7 pt,
    leftmargin = 20pt,
    rightmargin = 20pt,
    linecolor = darkgrey,
    topline = false,
    bottomline = false,
    rightline = false,
    footnoteinside = false
]{nttn}{Notation}

\theoremstyle{remarkstyle}
\newtheorem{rmk}{Remark}

% tikzcd
% % Substituting symbols for arrows in tikz comm-diagrams.
\tikzset{
  symbol/.style={
    draw=none,
    every to/.append style={
      edge node={node [sloped, allow upside down, auto=false]{$#1$}}}
  }
}


\begin{document}
\title{Number Theory Notes}
\author{JH}
\date{Date}
\maketitle

\tableofcontents

\chapter{Finite Fields}
\section{Generalities}
\subsection{Finite fields}
\begin{dfn}[Characteristic of a field]
    If $K$ is a field then the map $\Z \to K$ induced by $1 \mapsto 1$
    is a ring morphism.
    The image of this morphism is an integral domain since $K$ is a field,
    hence the kernel is a prime ideal. 
    Since $\Z$ is a PID, 
    we can define the characteristic of $K$, denoted
    $\Char{K}$ to be the positive generator of the kernel.
    \footnote{A foot}
\end{dfn}

\begin{prop}[Frobenius map]
    \link{frobenius}
    If $K$ is a field and $\Char{K}$ is prime then 
    \[\si_p : K \to K \quad := \quad x \mapsto x^p\]
    is an injection.
\end{prop}
\begin{proof}
    Easy to show $\si_p(0) = 0, \Si_p (1) = 1$. 
    Also
    \[\si_p (ab) = (ab)^p = a^p b^p = \si_p(a) \si_p(b)\]
    \[\si_p (a +b) = (a + b)^p = a^p + b^p = \si(a) + \si(b)\]
    by expanding the binomial and noting that when $1 \le k \le p$, 
    $ p \mid \binom{p}{k} k! (p-k)!$ and is coprime to the latter two,
    thus $p \mid \binom{p}{k}$.
    Since $\si_p$ is a morphism of fields it is injective.
\end{proof}

\begin{prop}[Classification of finite fields]
    Let $K$ be a finite field and 
    suppose $\Om \models \mathrm{ACF_p}$ where $p$ is prime
    and $q$ is a non-trivial power of $p$.
    Then 
    \begin{enumerate}
        \item $\Char{K} \ne 0$ and $\abs{K} = p^{[K : \F_p]}$
        \item $\F_q := \set{x \in \Om \st x^{q} = x}$ is the unique 
            subfield of $\Om$ with $q$ elements.
        \item If $\abs{K} = q$ then $K \iso \F_q$.
    \end{enumerate}
\end{prop}
\begin{proof}~
    \begin{enumerate}
        \item If $\Char{K} = 0$ then $\Z$ injects into $K$ thus 
            thus $\aleph_0 \le \abs{\Z} \le \abs{K}$ which is false.
            Since $[K : \F_p]$ is the cardinality of any basis $B$ of $K$
            as a vector space over $\F_p$ and $K \iso {\F_p}^B$,
            $\abs{K} = \abs{{\F_p}^B} = p^{[K : \F_p]}$.
        \item Easy to show elementarily that $\F_q$ is a subfield.
            As polynomials over a field are seperable if and only
            the gcd of the derivative and the polynomial is $1$, 
            \[ D(X^q - X) = q X^{q - 1} - 1 = - 1 \]\
            Hence it has $q$ distinct roots in the algebraic closure of $\Om$,
            namely $\Om$ itself. 
            Hence $\abs{\F_q} = q$.
            Uniqueness: if $L \le \Om$ and $\abs{L} = q$ 
            then for any unit $x \in L \setminus \set{0}$,
            $x^{q - 1} = 1$ by Lagrange and so $x \in \F_q$.
            Thus $L \subs \F_q$ and they have equal finite cardinality,
            so $L = \F_q$.
        \item If $L$ is a field such that $\abs{L} = q$ then
             the image of $\Z$ in $L$ has cardinality dividing $q$
             by Lagrange.
             Hence $\Char{L} = p$ and the image of $\Z$ is $\F_p$.
             Finitely generate $L$ over $\F_p$
             and for each generator $a$ the minimal polynomial of $a$
             over $\F_p$ splits in $\Om$ since it is aglebraically closed.
             By `embedding finite extensions via conjugates' in Galois Theory,
             there is a map $L \to F_q$ which is injective.
             It is an isomorphism since they have the same finite cardinality.
    \end{enumerate}
\end{proof}

\subsection{Multiplicative group of a finite field}
\begin{dfn}[Euler's Totient Function]
    If $1 \le a \le d$ in $\Z$ then $a$ is coprime to $d$
    if and only if $\bar{a} \in \Z / d \Z$ is a generator since
    \begin{align*}
            & (a,d) = 1 \\
        \iff \quad & \exists \la, \mu \in \Z, \la a + \mu d = 1\\
        \iff \quad & \exists \la \in \Z, \bar{\la a} = 1\\
        \iff \quad & \< \bar{a} \> = \Z / d \Z
    \end{align*}
    We define Euler's totient function 
    \[
        \phi(d) := \abs{\set{a \in \Z / d \Z \st \<a\> = \Z / d \Z}} = 
        \abs{\set{a \in \Z \st 1 \le a \le d \AND (a,d) = 1}}
    \]
\end{dfn}

\begin{nttn}
    For any cyclic group $G$, 
    let $\Phi(G) = \{g \in G \st \<g\> = G\}$ 
    be the set of generators.
\end{nttn}

\begin{prop}[Partitioning cyclic groups]
    If $n \in \Z_{>0}$ then $n = \sum_{d \mid n} \phi(d)$.
\end{prop}
\begin{proof}
    Let $n \in \Z_{>0}$ and let $d$ divide $n$. 
    Then by some cyclic group theory there exists a unique cyclic subgroup 
    $C_d \le \Z /n \Z$ with cardinality $d$.
    We want to show that $\Z / n \Z = \bigsqcup_{d \mid n} \Phi(C_d)$.
    Indeed if $x \in \Z / n \Z$ then $\<x\>$ has some order $d$ 
    dividing $n$ by Lagrange.
    Hence $x \in \Phi(\<x\>) = \Phi(C_d)$.
    Thus $\Z / n\Z \subs \cup_{d \mid n} \Phi(C_d)$.

    To show it is disjoint notice that if $x$ is in 
    $\Phi(C_d) \cap \Phi (C_e)$ then 
    $d$ and $e$ are both the order of $x$.
\end{proof}

\begin{prop}[Sufficient condition for cyclic]
    Let $G$ be a group such that for any $d \mid \abs{G}$,
    \[\abs{\set{x \in G \st x^d = e}} \le d \]
    then $G$ is cyclic.
\end{prop}
\begin{proof}
    We show that for all divisors of $\abs{G}$ there is an element
    of $G$ of that order. 
    Then in particular $\abs{G} \mid \abs{G}$ 
    and so there is a generator of $G$.

    Let $d \mid G$. 
    Consider $\set{x \in G \st x \text{ has order } d}$.
    If it is non-empty, then take such an $x$:
    \[
        \<x\> \subs
        \set{g \in G \st g^d = e}
    \]
    and so $d \le \abs{\<x\>} \le \abs{\set{g \in G \st g^d = e}} \le d$.
    Then $\<x\> = \set{g \in G \st g^d = e}$.
    Hence for $g \in G$,
    \begin{align*}
        g \text{ has order } d &\iff
         g \text{ has order } d \AND g^d = e\\
         &\iff g \text{ has order } d \AND g \in \<x\>\\
         &\iff \<g\> = \<x\>
    \end{align*}
    Hence $\abs{\set{x \in G \st x \text{ has order } d}} = \phi(d)$
    In either case, (empty or not), 
    $\abs{\set{x \in G \st x \text{ has order } d}} \le \phi(d)$
    
    Assume for a contradiction that there exists a $d$ such that 
    $\set{x \in G \st x \text{ has order } d}$
    is empty.
    Then partitioning 
    \[G = \bigsqcup_{d \;\mid \;\abs{G}} \set{x \in G \st x \text{ has order } d}\]
    we have that 
    \[
        \abs{G} = \sum_{d \mid \abs{G}} 
            \abs{\set{x \in G \st x \text{ has order } d}}
            < \sum_{d \mid \abs{G}} \phi(d) 
            = \abs{G}
    \]
    a contradiction.
\end{proof}

\begin{prop}[$\F_q^*$ is cyclic]
    \link{fin_field_units_cyclic}
    Suppose $d \mid \abs{\F_q^*}$.
    Then since $\F_q[X]$ has division algorithm,
    \[\abs{\set{x \in \F_q^* \st x^d = 1}} \le d \]
    Hence $\F_q^*$ is cyclic.
\end{prop}

\section{Equations over a finite field}
\begin{prop}{Power sums lemma}
    \link{power_sum}
    Let $u \in \N$ and $K$ be field with $|K| = q$ 
    a power of a non-trivial prime.
    Then 
    \[
        \sum_{x \in K} x^u = 
        \begin{cases}
            -1 &, 1 \le u \AND q - 1 \mid u\\
            0 &, \text{ otherwise}
        \end{cases}
    \]
\end{prop}
\begin{proof}
    Case $u = 0$ then 
    $\sum_{x \in K^n} x^u = \sum_{x \in K^n} 0 = 0$.

    Case $1 \le u \AND q - 1 \mid u$ then for some $d$,
    \[\sum_{x \in K} x^u = \sum_{x \in K} (x^(q-1))^d = 
     = \sum_{x \in K^*} 1^d = (q - 1) 1 = -1\]

    Case $1 \le u \AND q - 1 \nmid u$ then there exist 
    $d, r \in \N$ such that $u = (q-1)d + r$ and $0 < r < q - 1$.
    Let $y$ be a generator of $K^*$ 
    (\linkto{fin_field_units_cyclic}{$K^*$ is cyclic}).
    Then suppose for a contradiction that $y^u = 1$,
    then $q - 1 \mid u$ since $q - 1$ is the order of $y$, 
    a contradiction.
    Multiplying by $y$ is a bijection on the group, 
    hence
    \[
        \sum_{x \in K^n} x^u = \sum_{x \in K^n} (yx)^u = 
        y^u \sum_{x \in K^n} x^u
    \]
    Thus $(1 - y^u) \sum_{x \in K^n} x^u = 0$ and so 
    $\sum_{x \in K^n} x^u = 0$, 
    as $y^u \ne 1$.
\end{proof}

\begin{dfn}[Vanishing]
    Let $R$ be a ring.
    Suppose for all $I \subs R[x1, \dots , x_n]$
    We define the vanishing of $I$ in $R$, 
    \[\V(I,R) := \{x \in R^n \st \forall f \in I, f(x) = 0\}\]
    If the context is obvious we just write $\V(I)$.
\end{dfn}

\begin{prop}[Chevalley]
    Suppose for all $f \in I \subs K[x_1, \dots , x_n]$ (finite), 
    \[\sum_{f \in I} \deg f < n\]
    Then 
    \[\abs{\V(I)} \ontop{p}{=} 0\]
\end{prop}
\begin{proof}
    Consider $P := \prod_{f \in I} (1 - f^{q-1})$.
    This is well defined as $I$ is finite.
    We show that $\V(I) = P^{-1} (1)$.
    Let $x \in K^n$.
    \[
        x \in \V(I) \implies \forall f \in I, f(x) = 0 
        \implies  {f(x)}^{q - 1} = 0 
        \implies P(x) = 1
    \]
    \[
        x \notin \V \implies
        \exists f \in I, f \ne 0
        \implies {f(x)}^{q - 1} = 1
        \implies P(x) = 0
    \]
    Let 
    $S : K[x_1 ,\dots, x_n] \to K := f \to \sum_{x \in K^n} f(x)$
    Then $S(P) = \sum_{x \in V(I)} 1 \ontop{p}{=} \abs{\V(I)}$.
    Thus we need show that $S(P) = 0$.

    \[\deg P = \sum_{f \in I} (q - 1)\deg f 
    = (q - 1) \sum_{f \in I} \deg f < n \implies < (q - 1) n\]
    by assumption.
    Hence there exists a finite set $T$ and $\la_i \in K$ such that
    \[P = \sum_{i \in T} \la_i \prod_{j = 1}^n x_j^{u_{ij}}\]
    and for all $i \in T$, $\sum_{j = 1}^n u_{ij} < (q - 1) n$.
    Then 
    \begin{align}
        S(P) &= \sum_{x \in K^n} P(x)\\
            &= \sum_{x\in K^n} \sum_{i \in T} \la_i 
            \prod_{j = 1}^n x_j^{u_{ij}}\\
            &= \sum_{i \in T} \la_i \sum_{x\in K^n} 
            \prod_{j = 1}^n x_j^{u_{ij}}\\
    \end{align}
    Let $i \in T$ then there exists a $k$ such that 
    $u_{ik} < q - 1$ so
    \begin{align}
        & \sum_{x\in K^n} \prod_{j = 1}^n x_j^{u_{ij}}\\
        = & \sum_{x_1 \in K} \cdots \sum_{x_n \in K} 
        \prod_{j = 1}^n x_j^{u_{ij}}\\
        = & \sum_{x_1 \in K} \cdots \cancel{\sum_{x_k \in K}} \cdots \sum_{x_n \in K} 
        \prod_{j \ne k} x_j^{u_{ij}} \sum_{x_k \in K} x_k^{u_{ik}}\\
        = & \sum_{x_1 \in K} \cdots \cancel{\sum_{x_k \in K}} \cdots \sum_{x_n \in K} 
        \prod_{j \ne k} x_j^{u_{ij}} 0
    \end{align}
    The last part using the \linkto{power_sum}{power sum lemma}.
    Hence $\abs{\V(I)} \ontop{p}{=} S(P) = 0$
\end{proof}

\begin{cor}[Non-trivial vanishing]
    Suppose for all $f \in I \subs K[x_1, \dots , x_n]$ (finite), 
    \[\sum_{f \in I} \deg f < n\]
    and $0 \in \V(I)$ then $\exists x \in \V(I) \setminus \set{0}$.
\end{cor}
\begin{proof}
    If $\abs{V} = 1$ then $p \not \mid \abs{\V}$ which is a contradiction.
    Thus the vanishing is non-trivial.
\end{proof}

\begin{dfn}[Homogeneous]
    $f \in K[x_1, \dots , x_n]$ is homogeneous with degree $m$ 
    if all monomials are of degree $m$.
\end{dfn}

\begin{cor}[Conics over a finite field]
    If $3 \le n$ then if $f \in K[x_1, \dots , x_n]$ 
    is homogeneous with degree $2$ then it has a non-trivial zero.
\end{cor}

\section{Quadratic reciprocity}
\begin{prop}[Exact sequence]
    If $K$ is a finite field,
    \begin{itemize}
        \item If $\Char{K} = 2$ then all elements are square.
        \item If $\Char{K} \ne 2$ 
        then the non-zero squares form a subgroup of index $2$,
        and is the kernel of the group morphism 
        $x \to x^{\frac{q-1}{2}}$ into $\<-1\>$.
    \end{itemize}
    I can't be bothered to make the exact sequence.%
\end{prop}
\begin{proof}~
    \begin{itemize}
        \item If $\Char{K} = 2$ then the 
        \linkto{frobenius}{Frobenius map} $\si_2 : x \mapsto x^2$
        is an automorphism of $K$. 
        Hence the preimage of any element squares to that element.
        \item If $\Char{K} \ne 2$ then generate $K^* = \<g\>$ 
        since it is cyclic.
        The map $x \to x^{\frac{q-1}{2}}$ 
        has kernel $\set{x \in K \st x \text{ square}}$ since
        (writing any element as a multiple of $g$)
        \[
            g^n \in \ker \iff g^\frac{n(q-1)}{2} = 1 \iff 
            q - 1 \mid \frac{n(q - 1)}{2} \iff n \text{ even} 
            \iff x \text{ square}
        \]
        We check where the generator $g$ is sent. 
        If $g^{\frac{q - 1}{2}} = 1$ then the order of $g$ 
        is less than $q - 1$ which is a contradiction
        hence the image is non-trivial.
        Any element of the image of the map squares to $1$
        hence solves $x^2 - 1 = 0$,
        which only has two solutions in $K$.
        Thus the image is $\<-1\>$ and the index of the kernel is $2$.
    \end{itemize}
\end{proof}

\begin{dfn}[Legendre symbol]
    If $p$ is prime that is not $2$ and $x \in \F_p$ then
    \[\legen{x}{p} := 
    \begin{cases}
        x^{\frac{p-1}{2}} &, x \text{ unit}\\
        0 &, x = 0
    \end{cases}\]
    Check that for each $p$ this is a homomorphism $\F_p \to \<-1\>$.
\end{dfn}

\begin{dfn}[$\ep(n)$]
    If $n \in \Z$ is odd 
    \[\ep(n):=\frac{n-1}{2} (\mathrm{mod} 2)\]
\end{dfn}

\begin{prop}[Computations]
    \begin{align*}
        \legen{1}{p} =& 1\\
        \legen{-1}{p} =& (-1)^{\ep(p)}
    \end{align*}
\end{prop}

\begin{prop}[Quadratic reciprocity]
    Let $l \ne p$ be primes that aren't $2$.
    Then \[\legen{l}{p}\legen{p}{l} = (-1)^{\ep(l)\ep(p)}\]
\end{prop}
\begin{proof}
    Let $w$ be order $l$ element of $\Om$,
    the algebraic closure of $\F_p$.
    For $x \in \F_l$ write $w^x$ to be $w^r$ for any $r \in \Z$ such that
    $x = \bar{r} \in \F_l$ (independant of choice of $r$ by $w^l = 1$).
    Let 
    \[y = \sum_{x\in \F_l} \legen{x}{l} w^x \in \Om\]

    We first show that $y^2 = (-1)^{\ep(l)} \bar{l}$, 
    where $\bar{l} \in \F_p$.
    \begin{align*}
        y^2 =& 
        (\sum_{x \in \F_l} \legen{x}{l} w^x)
        (\sum_{y \in \F_l} \legen{y}{l} w^y)\\
        =& \sum_{x \in \F_l} \sum_{y \in \F_l} 
        \legen{x}{l} w^x \legen{y}{l} w^y\\
        =& \sum_{x \in \F_l} \sum_{y \in \F_l} \legen{xy}{l} w^{x+y}\\
        =& \sum_{u \in \F_l} \sum_{x \in \F_l} \legen{x(u - x)}{l} w^u
    \end{align*}
    Case on what $x$ is:
    \begin{align*}
        x \ne 0 \implies& \legen{x(u - x)}{l} =& \legen{xu - x^2}{l}\\
            & =& \legen{x^2}{l} \legen{-1}{l} \legen{1-\frac{u}{x}}{l}\\
            & =& x^{p - 1} \legen{-1}{l} \legen{1 - \frac{u}{x}}{l}\\
            & =& (-1)^{\ep(l)} \legen{1 - \frac{u}{x}}{l}
    \end{align*}
    If $x = 0$ then clearly $\legen{x(u - x)}{l} = 0$.
    Hence 
    \[
        y^2 = \sum_{u \in \F_l} \sum_{x \in \F_l^*} 
        (-1)^{\ep(l)} \legen{1 - \frac{u}{x}}{l}
        = (-1)^{\ep(l)} \sum_{u \in \F_l}  \sum_{x \in \F_l^*}
        \legen{1 - \frac{u}{x}}{l}
    \]
    Given $x \ne 0$, case on what $u$ is:
    \begin{align*}
        u = 0 \implies &\sum_{x \in \F_l^*} \legen{1 - \frac{u}{x}}{l} \\
        = &\sum_{x \in \F_l^*} \legen{1}{l} \\
        = &\sum_{x \in \F_l^*} 1\\
        = & \bar{l} -1
    \end{align*}
    \begin{align*}
        u \ne 0 \implies &\sum_{x \in \F_l^*} \legen{1 - \frac{u}{x}}{l}\\
            = &\sum_{x \in F_l^*} \legen{1 - \frac{1}{x}}{l}\\
            = &\sum_{s \in \F_l^*} \legen{1 - s}{l}\\
            = &\sum_{s \in \F_l \setminus \set{1}} \legen{s}{l}\\
            = &\sum_{s \in \F_l} \legen{s}{l} - \legen{1}{l}\\
            = & - 1 
    \end{align*}
    Since the index of the kernel of $\legen{\star}{l}$ is $2$,
    and the cosets have equal cardinality.
    Hence 
    \begin{align*}
        y^2 (-1)^{\ep(l)} &= \sum_{u \in \F_l}  \sum_{x \in \F_l^*}
            \legen{1 - \frac{u}{x}}{l}\\
            &= \bar{l} - 1 - \sum_{u \in \F_l^*} w^u\\
            &= \bar{l} - (1 + w + w^2 + \dots + w^l)
    \end{align*}
    since $l$ is prime. 
    Note that $0 = w^l - 1 = (w+1)(1+w + \dots + w^l)$.
    Hence $1+w + \dots + w^l = 0$ and $y^2 = (-1)^{\ep(l)} \bar{l}$.

    Next we show that $y^{p - 1} = \legen{p^-1}{l}$.
    \begin{align*}
        y^p = & \sum_{x \in \F_l} {\legen{x}{l}}^p w^xp & \text{ `Freshman's dream'}\\
            = & \sum_{x \in \F_l} {\legen{x}{l}} w^xp & 
            \legen{x}{l}= \pm 1 \text{ and } p \text{ is odd}\\
            = & \sum_{z \in \F_l} \legen{z p^{-1}}{l} w^z\\
            = & \legen{p^{-1}}{l}(\sum_{z \in \F_l} \legen{z}{l} w^z)\\
            = & \legen{p^{-1}}{l} y
    \end{align*}
    Hence \[y^{p - 1} = \legen{p^{-1}}{l} = (\legen{p{l}})^{-1}\]
    thus 
    \begin{align*}
        \legen{l}{p}\legen{p}{l} =& \legen{l}{p} y^{1 - p}\\
        =& \legen{l}{p} (y^2)^{\frac{1 - p}{2}}\\
        =& \legen{l}{p} ((-1)^{\ep(l)} \bar{l})^{\frac{1 - p}{2}}\\
        =& \legen{l}{p} (\legen{(-1)^{\ep(l)}l}{p})^{-1}\\
        =& (\legen{(-1)^{\ep(l)}}{p})^{-1}\\
        =& (-1)^{\ep(l)\ep(p)}
    \end{align*}
\end{proof}
\chapter{p-adic Fields}
\section{p-adic Integers and Rationals}
\begin{dfn}[Projective system]
    Let $\CC$ be a category.
    A contravariant functor $F : (\N,\le) \to \CC$ 
    is called a projective system.
\end{dfn}

\begin{dfn}[Projective system $A$]
    Define a contravariant functor $A : (\N,\le) \to \RING$ 
    such that for each $n$
    \[A_n := \Z / p^n \Z \quad \text{ and } 
    \quad \pi_n : \Z \to A_n \text{ is the projection}\]
    and for any $n$ such that $1 \le n$, 
    there exists a surjective ring morphism 
    $\phi_n : A_n \to A_{n-1}$ such that
    $\phi_n \circ \pi_n = \pi_{n-1}$ and 
    $\ker(\phi_n) = p^{n-1} A_n$.
\end{dfn}
\begin{ex}
    Check that such a $\phi_n$ exists.
\end{ex}

\begin{dfn}[$p$-adic integers]
    Let 
    \[\Z_p = \{x \in \prod_{n \in \N} A_n \st 
    (\forall n \in \N, x_n \in A_n) \AND
    (\forall n > 0, \phi_n(x_n) = x_{n-1})\}\]
    be the projective limit.
    Define addition and multiplication pointwise. 
    Verify that this $\Z_p$ is a ring with $0 = (0)_{n \in \N}$
    and $1 = (1)_{n \in \N}$.

    For each $n \in \N$ let $\ep_n : \Z_p \to A_n$ be the 
    ring morphisms mapping $x \mapsto x_n$.
    Note that by definition $\phi_n \circ \ep_n = \ep_{n-1}$.

    In addition,
    provide each $A_n$  with the discrete toplogy,
    giving $\prod_{n \in \N} A_n$ the product topology
    and $\Z_n$ the subset topology.
\end{dfn}

\begin{prop}[$\Z_p$ is compact]
    \link{Z_p_compact}
    Since each $A_n$ is finite, 
    each $A_n$ is compact.
    Hence by Tychonoff's theorem
    the product is compact.
    Since closed in compact is compact
    we just need to show that $\Z_p$ is closed.

    We want to write $\Z_p$ as the intersection of 
    closed sets 
    \[D_k:= \set{x \in \prod_{n\in\N} A_n \st \phi_k (x_k) = x_{k-1}}\]
    for $k \in \N$.
    Clearly 
    \[\bigcap_{k\in\N} D_k = \Z_p\]
    and 
    \[D_k = \bigcup_{x_{k-1} \in A_{k-1}} \brkt{\ep_{k-1}^{-1}(x_{k-1}) 
    \cap \bigcup 
    \set{\ep_k^{-1}(x_k) \st x_k \in A_k \AND \phi_k(x_k) = x_{k-1}}}\]
    Since each $\set{x_k}$ is closed in $A_k$, 
    each preimage $\ep_k^{-1}(x_k)$ is closed.
    Thus the finite union of the preimages
    \[\bigcup 
    \set{\ep_k^{-1}(x_k) \st x_k \in A_k \AND \phi_k(x_k) = x_{k-1}}\]
    is closed.
    Since each $\set{x_{k-1}}$ is closed in $A_{k-1}$, 
    each preimage $\ep_{k-1}^{-1}(x_{k-1})$ is closed.
    Thus intersection 
    \[\brkt{\ep_{k-1}^{-1}(x_{k-1}) 
    \cap \bigcup 
    \set{\ep_k^{-1}(x_k) \st x_k \in A_k \AND \phi_k(x_k) = x_{k-1}}}\]
    is closed.
    Hence the finite union is closed and $D_k$ is closed.
    Arbitrary intersection of closed is closed so $\Z_p$ 
    is closed and thus compact.
\end{prop}

\begin{prop}[Universal property of $\Z_p$]
    Suppose $R$ is a ring with ring morphisms 
    $\rho_n : R \to A_n$ for each $n \in \N$
    such that for each $n > 0$, 
    $\phi_n \circ \rho_n = \rho_{n-1}$.
    Then there exists a unique ring morphism $f : R \to \Z_p$
    such that for each $n$, 
    $\ep_n \circ f = \rho_n$.
\end{prop}
\begin{proof}
    If there exists such a map then it is unique:
    suppose $f,g$ both satisfy the given properties.
    Then for any $n$ and any $a \in R$
    $\ep_n \circ f (a)= \rho_n (a) = \ep_n \circ g (a)$.
    Thus $f(a) = g(a)$, by the property of products
    (if they agree on all the projections they are equal).

    For existance we let $a \in R$ and consider the set
    \[\bigcap_{n \in \N} {\ep_n^{-1} \circ \rho_n (a)}\]
    show that it has cardinality $1$,
    and let $f$ map $a$ to this unique element.
    If $x,y \in \bigcap_{n \in \N} {\ep_n^{-1} \circ \rho_n (a)}$
    then for any $n \in \N$, 
    $\ep_n (x) = \rho_n (a) = \ep_n (y)$.
    Thus $a = b$ by the property of products.
    Hence the cardinality is $\le 1$.

    To show that the set is non-empty,
    take $x = (\rho_n (a))_{n \in \N}$.
    This is in $\Z_p$ since for each $n>0$,
    $\phi_n \circ \rho_n(a) = \rho_{n - 1}(a)$.
    Also it is in the intersection since for each $n$,
    $\ep_n(x) = \rho_n(a)$. 
    Hence the cardinality is $1$.
    Hence $f$ is well-defined and for all $n \in \N$, 
    $\ep_n \circ f = \rho_n$.

    For any $n$, 
    \[\ep_n \circ f (a + b) = \rho_n (a + b) = \rho_n (a) + \rho_n (b)
    = \ep_n \circ f(a) + \ep_n \circ f(b) = \ep_n (f(a) + f(b))\]
    Hence by property of products $f(a + b) = f(a) + f(b)$
    and similarly for multiplication.
    Note that for any $n$, 
    $\ep_n \circ f(1) = \rho_n (1) = 1$.
    Hence $f(1) = 1$.
    Thus $f$ is a ring morphism.
\end{proof}

\begin{cor}[$\Z$ injects into $\Z_p$]
    Then there exists a unique injective ring morphism 
    $\io : \Z \to \Z_p$
    such that for each $n$, 
    $\ep_n \circ \io = \pi_n$.
\end{cor}
\begin{proof}
    By the previous theorem the morphism exists and is unique.
    It must send $1 \mapsto 1$ hence $\io(x) = 0$ would imply
    $\pi_n (x) = \ep_n \circ \io(x) = 0$ for all $n \in \N$.
    Hence for any $n \in \N$, 
    $p^n \mid x$.
    Thus $x = 0$.
\end{proof}

\begin{prop}[Multiplying by $p^n$ is injective 
    and $x_n = 0$ implies $x \in p^n \Z_p$]
    \link{multiplying_p_n_injective}
    \begin{center}\begin{tikzcd}
        0 \ar[r] &\Z_p \ar[r, "p^n \cdot"] & 
        \Z_p \ar[r, "\ep_n"] & A_n \ar[r] & 0
    \end{tikzcd}\end{center}
    is a short exact sequence of abelian groups.
\end{prop}
\begin{proof}
    To check that the morphism $\Z_p \to \Z_p$ 
    multiplying by $p^n$ is injective
    it suffices to show that multiplying by $p$ is 
    injective.
    Suppose $x$ is in the kernel of this map, 
    then $px = 0$ thus for any $n$, 
    $p x_{n+1} = \ep_{n+1}(px) = 0$.
    We show that for any $n$, $x_n = 0$.
    There exists $a \in \Z$ such that
    $\pi_{n+1} (a)= x_{n+1}$.
    Since $\pi_{n+1}(pa) = p x_{n+1} = 0$,
    $pa = p^{n+1} b$ for some $b \in \Z$.
    Hence $a = p^n b$ since $\Z$ is an integral domain.
    Thus $\pi_n (a) = x_n = 0$.
    Thus $x = 0$.

    To check that the $p^n \Z_p = \ker(\ep_n)$
    we note that for any $x \in \Z_p$, 
    $\ep_n (p^n x) = p^n x_n = 0 \in A_n$.
    Hence $p^n \Z_p \subs \ker(\ep_n)$.
    For the other direction suppose $\ep_n(x)= 0$.
    Suppose $n \le m \in \Z$.
    Then there exists a unique $a_m \in \Z$ such that $0 \le a < p^m$
    and $\pi_m (a_m) = \ep_m(x)$.
    Then 
    \[\pi_n (a_m) = \phi_m \circ \cdots \circ \phi_{n+1} \pi_m (a_m) = 
    \phi_m \circ \cdots \circ \phi_{n+1} \ep_m (x) = \ep_n (x) = 0\]
    Thus there exists a unique $b_m \in \Z$ such that $a_m = p^n b_m$.
    
    Let $b = (\pi_m(b_m))_{m \in \N} \in \Z_p$.
    Note that multiplying by $p^n$ commutes with all the map
    as they are ring homomorphisms.
    Then for any $m \in \N$,
    \begin{align*}
        \phi_{m+1} \ep_{m+1}(b) &= \phi{m+1} \circ \pi_{m+1} (b_{m+1})
        &= \phi{m+1} \circ \pi_{m+1} (p^n a_{m+1})\\
        &= p^n \phi_{m+1} \circ \pi_{m+1} (p^n a_{m+1})
        &= p^n \pi_m (a_m) \\
        &= \pi_m(b_m) &= \ep_m(b)
    \end{align*}
    Hence $b \in \Z_p$.
    Furthermore, 
    let $m \in \N$ then
    \[ \ep_m(p^n b) = p^n \pi_m (b_m) = \pi_m (p^n b_m) = 
    \pi_m (a_m) = \ep_m(x)\]
    Hence $p^n b = x$.
    Thus $x \in p^n \Z_p$.
\end{proof}

\begin{prop}[$\Z_p$ is a local ring, decomposition of non-zero elements]
    \link{Z_p_non_zero_decomp}
    If $x \in \Z_p$ then
    \begin{enumerate}
        \item $x_n \in A_n$ is a unit if and only if $x_n \notin p A_n$.
        \item $x \in \Z_p$ is a unit if and only if $x \notin p \Z_n$.
        \item For any $ x \in \Z_p \setminus \set{0}$ there exist 
            unique $n \in \N$ and $u \in \Z_p$ such that
            $u$ is a unit and $p^n u = x$.
    \end{enumerate}
\end{prop}
\begin{proof}~
    \begin{enumerate}
        \item If $x_n$ is a unit and $x_n \in p A_n$ then
            write $x_n = p y_n $ for $y_n \in A_n$.
            We see that $p$ is a unit since $x_n ^{-1} p y_n = 1$.
            However $p$ is nilpotent since $p^n = 0$ a contradiction.
            Hence $x_n \notin p A_n$.
            Conversely if $x_n \notin p A_n$ then
            supposing $x_1 = 0$ deduces $x \in p \Z_p$ by 
            the \linkto{multiplying_p_n_injective}{
                previous proposition}.
            Hence $x_n \in p A$ a contradiction.
            Thus $x_1 \ne 0 \in A_1$, a field,
            so $x_1$ is a unit in $A_1$.
            Hence there exist $x_\Z,y_\Z,z_\Z \in \Z$
            such that $\io(x_\Z) = x$ and 
            \begin{align*}
                x_\Z y_\Z + p z_\Z =& 1 \\
                \implies \pi_n(x_\Z y_\Z + p z_\Z) =& 1\\
                \implies x_n y_n + p z_n =& 1\\
                \implies x_n y_n (1 + \dots + (p z_n)^{n-1}) &= 
                1 - (pn)^z = 1 \in A_n\\
                \implies x_n \text{ is a unit}
            \end{align*}
            Hence $x_n$ is a unit if and only if $x_n \notin p A_n$.
        \item If $x$ is a unit of $\Z_p$
            then in particular $x_1$ is a unit.
            Suppose $x \in p \Z_p$ then $x_1 = 0$ by the 
            \linkto{multiplying_p_n_injective}{previous proposition}.
            Hence $x_1$ is not a unit, a contradiction.
            Thus $x \notin p \Z_p$.
            
            For the converse suppose $x \notin p \Z_p$
            then by the
            \linkto{multiplying_p_n_injective}{previous proposition}
            $x_1 \ne 0$.
            For any $n \in \N$, 
            if $x_n \in A_n$ then 
            $x_1 = \phi_n \circ \cdots \phi_2 x_n = 0$
            which is false.
            Hence for any $n \in \N$, $x_n \notin p A_n$
            which by the first part implies there exists
            a unique $y_n \in A_n$, $x_n y_n = 1$.
            We show that $y := (y_n)_{n \in \N}$ is the inverse of $x$
            in $\Z_p$.
            To show that $y \in \Z_p$ let $n \in \N$.
            \[x_n \phi_{n+1} (y_{n+1}) = 
            \phi_{n+1}(x_{n+1})\phi_{n+1}(y_{n+1})
            \phi_{n+1}(x_{n+1} y_{n+1}) = \phi(1) = 1\]
            Hence $\phi_{n+1} (y_{n+1}) = y_n$ 
            by uniqueness of inverses in $A_n$.
            To show that $x y = 1$ note that
            for any $n \in \N$, 
            $\ep_n(xy) = x_n x_y = 1$.
            Hence $xy = 1$.
        \item Let $x \in \Z_p$ be non-zero and consider
            the set \[\set{n \in \N \st \ep_n (x) = 0}\]
            This is non-empty since $\ep_0 (x) = 0$.
            By induction there exists a maximum of this set,
            call this $n$.
            Since $\ep_n(x) = 0$ by the 
            \linkto{multiplying_p_n_injective}{previous proposition}
            $x = p^n y$ for some $y \in \Z_p$.
            Suppose $y \in p \Z_p$ then $\ep_{n + 1} (x) = 0$
            which is a contradiction with maximality.
            Hence by the previous part of this proposition
            $y$ is a unit.
            
            Suppose we have another decomposition 
            $x = p^m z$ with $z$ a unit.
            Then by maximality of $n$, $m \le n$.
            By the 
            \linkto{multiplying_p_n_injective}{previous proposition}
            we have that multiplication by $p^m$ is injective.
            Hence $p^n y = p^m z$ implies $p^{n-m} y = z$.
            Since $z$ is a unit, $n - m = 0$.
            Hence $n = m$ and $y = p^{n-m} y = z$.
    \end{enumerate}
\end{proof}

\begin{dfn}[$\N_\infty$]
    On the set $\N_{\infty} := \N \cup \set{\infty}$ 
    define commutative addition such that
    if $n,m \in \N$ then it is the usual addition and
    for any $x \in \N_\infty$, $x + \infty = \infty$.
    We order the set using $\le$, 
    where it is the usual $m\le n$ for $m,n \in \N$
    and for any $x \in \N_\infty$, $x \le \infty$ and 
    if $\infty \le x$ then $x = \infty$.
    This is a total order hence we have a well defined
    infimum for any non-empty set.
\end{dfn}

\begin{dfn}[$p$-adic valuation]
    Given $p$ a prime, define $v_p : \Z_p \to \N_\infty$
    sending any non-zero $x$ to $n$,
    where $n \in \N$
    and $u \in \Z_p$ is a unit such that $x = p^n u$.
    In the other case we define $v_p(0):=\infty$.
\end{dfn}

\begin{prop}
    For any $p$ prime and $x,y \in \Z_p$
    \[v_p(x y) = v_p(x) + v_p(y), \quad 
    \inf \set{v_p(x), v_p(y)} \le v_p(x + y)\]
\end{prop}
\begin{proof}
    Case on what $x,y$ are.
\end{proof}

\begin{cor}
    \link{Z_p_int_dom}
    $\Z_p$ is an integral domain.
\end{cor}
\begin{proof}
    Let $x, y \in \Z_p$ be such that $xy = 0$.
    Suppose for a contradiction both $x, y$ are non-zero.
    Then $v_p(x), v_p(y) \in \N$ hence
    $\infty = v_p(x y) = v_p(x) + v_p(y) \in \N$, 
    a contradiction.
\end{proof}

\begin{dfn}[Metric on $\Z_p$]
    Define a norm on $\Z_p$ by 
    \[\abs{\star} : \Z_p \to \R_{\ge 0} := x \mapsto 
    \begin{cases}
        0 &, x = 0\\
        p^{-v_p(x)} &, x \ne 0
    \end{cases}\]
    This satisfies 
    \begin{enumerate}
        \item $\abs{x} = 0 \iff x = 0$
        \item $\abs{x + y} \le \max(\abs{x},\abs{y}) \le \abs{x} + \abs{y}$
        \item $\abs{xy} \le \abs{x} \abs{y}$
        \item $\abs{1} = 1$
    \end{enumerate}
    This induces a metric on $\Z_p$.
\end{dfn}
\begin{proof}
    Straight forward.
\end{proof}

\begin{prop}[Cosets are clopen balls]
    \link{cosets_are_clopen}
    For any $n$ and $a \in \Z$
    the coset $a + p^n \Z_p$ is a clopen ball
    $B_{\de} (a)$ for some $\de \in \R-{>0}$.
\end{prop}
\begin{proof}
    $b \in a + p^n \Z_p$ if and only if 
    $n \le v_p(b - a)$ if and only if 
    $\abs{b - a} \le p^{-n}$
    if and only if 
    $\abs{b - a} < \frac{p^{-n}+p^{1-n}}{2} =: \de$,
    as the image of the norm is discrete.
    Hence $a + p^n \Z_p = \bar{B_{p^{-n}(a)}} = B_\de(a)$ and is clopen.
\end{proof}

\begin{prop}[Induced topologies are equivalent]
    The metric topology $\TT_0$ is the same as the
    subspace topology $\TT_1$ from $\prod_{n \in \N} A_n$.
\end{prop}
\begin{proof}
    We first show that the neighbourhoods of points are the same.
    Call the neighbourhood filter for a point $a$ 
    in the metric tolopogy $N_0(a)$
    and the other $N_1(a)$.
    We use $\<\star \st \dots\>$ to mean the neighbourhood filter generated
    by $\{\star \st \dots\}$.
    \begin{align*}
        N_1(a) 
        &= \< U \cap \Z_p \st a \in U \in 
        \text{ product topology on $\prod A_n$}\>\\
        &=\< \ep_n^{-1} (U) \cap \Z_p \st \exists n \in \N, 
        a_n \in U \subs A_n\>\\
        &= \< U \subs \Z_p \st \exists n \in \N,
        a + \ker(\ep_n) \subs U\>\\
        &= \< U \subs \Z_p \st \exists n \in \N,
        a + p^n \Z_p \subs U\>\\
        &= \< U \subs \Z_p \st \exists \de > 0, B_\de(a) \subs U\>\\
        &= N_0(a)
    \end{align*}

    The penultimate equality is due to 
    \linkto{cosets_are_clopen}{cosets being clopen balls} for one 
    inclusion and
    the other inclusion follows from finding $n \in \N$
    such that $p^{-(n+1)} < \de < p^{-n}$.
    
    Since a subset $U$ is open in a topology
    if and only if for all points $a \in U, U \in N(p)$
    we see that $U \in \TT_0$ if and only if 
    $\forall p \in U, U \in N_0(p)$ if and only if
    $\forall p \in U, U \in N_1(p)$ if and only if
    $U \in \TT_1$.
\end{proof}

\begin{prop}[Topological properties of $\Z_p$]
    \link{Z_dense_in_Z_p}
    $\Z_p$ is complete in the topological sense
    and the image of $\Z$ is dense in $\Z_p$.
\end{prop}
\begin{proof}
    Any Cauchy sequence in $\Z_p$ has a
    subsequence converging to $x \in \Z_p$
    as $\Z_p$ is a \linkto{Z_p_compact}{compact} metric space.
    This is also the unique limit of the original sequence 
    as it is Cauchy.
    Hence $\Z_p$ is complete.

    Clearly $\bar{\io(\Z)} \subs \Z_p$.
    Let $x \in \Z_p$. 
    We want to show that there exists a sequence
    in $\io(\Z)$ converging to $x$, 
    hence showing that $x \in \bar{\io(\Z)}$.
    For any $n \in \N$ there exists an element 
    $b \in \Z$ such that $\pi_n(b) = \ep_n(x)$.
    Define the sequence $y: \N \to \Z_p := n \to \io(b)$.
    Then we claim that $\lim_{n \in \N} y(n) = x$
    Let $\de \in \R_{>0}$. 
    There exists $N \in \N$ such that $p^{-N}<\de$.
    Let $n \in \N$ be such that $N \le n$.
    Then $\ep_n(x - y(n)) = 0$ 
    \linkto{multiplying_p_n_injective}{implies}
    $x - y(n) \in p^n A_n$  
    and so 
    \[\abs{x - y(n)} = p^{-v_p(x - y(n))} \le p^{-n}
    \le p^{-N} < \de\]
    Thus the limit exists and is $x$.
    Hence $\bar{\io(\Z)} = \Z_p$.
\end{proof}

\begin{dfn}[$\Q_p$]
    Since $\Z_p$ is an 
    \linkto{Z_p_int_dom}{integral domain},
    we can construct its field of fractions.
    We call this $\Q_p$.
\end{dfn}

\begin{prop}[Inclusions into $\Q_p$]
    There is a unique injective ring morphism $\Z_p \to \Q_p$
    which (without confusion) we treat as $\subs$
    and there is a unique injective extension of the ring morphism 
    $\io : \Z \to \Z_p$ to $\Q \to \Q_p$.
    \begin{center}
    \begin{tikzcd}
        \Z \ar[r, "\subs"] \ar[d, "\io", swap]
        &\Q \ar[d, dashed]\\
        \Z_p \ar[r, "\subs", swap]&\Q_p
    \end{tikzcd}
    \end{center}
\end{prop}
\begin{proof}
    The inclusion $\Z_p \to \Q_p$ is a result
    of the construction of the field of fractions.
    We extend $\io$ by mapping $\frac{a}{b} \in \Q$ to
    $\frac{\io(a)}{\io(b)} \in \Q_p$.
    Check that it is well-defined and injective,
    a ring morphism and that the diagram above commutes.
\end{proof}

\begin{prop}
    $\Q_p \iso \Z_p[\frac{1}{p}]$ canonically and any unit of $\Q_p$
    can be uniquely written in the form $p^n u$ for $n \in \Z$
    and $u$ a unit in the image of $\Z_p$ under the isomorphism. 
\end{prop}
\begin{proof}
    Let $f : \Z_p [\frac{1}{p}] \to \Q_p$ such that 
    $\sum_{i = 0}^{n} x_i (\frac{1}{p})^i
    \mapsto \sum_{i = 0}^{n} \frac{x_i}{p^i}$.
    Clearly $f$ is well defined and injective.
    To show that it is surjective note that for any element
    $\frac{a}{b} \in \Q_p$ with $a, b \in \Z_p, b \ne 0$
    we can write \linkto{Z_p_non_zero_decomp}{$b = p^n u$} 
    for unique $n \in \N$ and $u$ a unit.
    Hence 
    $\frac{a}{b} = \frac{a}{p^n u} = \frac{a u^{-1}}{p^n}$
    which is due to an element of $\Z_p[\frac{1}{p}]$ via $f$.

    The same trick gives us the decomposition of units in $\Q_p$.
\end{proof}

\begin{dfn}[$p$-adic valuation for $\Q_p$]
    Extend the definition of $v_p$ to $\Q_p$
    by taking $x \ne 0$ to $n$ such that $p^n u = x$.

    Note that $0 \le v_p(x)$ if an only if 
    $x$ is a $p$-adic integer.
\end{dfn}

\begin{dfn}[Addition is a homeomorphism on $\Q_p$]
    Let $a \in \Q_p$.
    Then the map $\Q_p \to \Q_p$ sending 
    $b \mapsto a + b$ is a homeomorphism.
\end{dfn}
\begin{proof}
    Let $b \in \Q_b$ and let $\de \in \R_{>0}$.
    It suffices that $a + B_\de (b) \subs B_\de (a + b)$.
    Indeed if $c \in B_\de (b)$ then
    $\abs{a + c - (a + b)} = \abs{c - b} < \de$.

    This map has inverse $-a$ which is continuous for the 
    same reasons. 
    Hence $a+\star$ is a homeomorphism.
\end{proof}

\begin{prop}[Topological properties of $\Q_p$]
    Useful properties:
    \begin{enumerate}
        \item For any $n \in \N$, 
        $p^n \Z_p$ is clopen in $\Q_p$,
        in particular $\Z_p$ is open in $\Q_p$.
        \item $\Q_p$ is locally compact
        and $\io(\Q)$ is dense in $\Q_p$.
        \item $\Q_p$ is complete.
    \end{enumerate}
\end{prop}
\begin{proof}
    Since $\Z_p$ and $\Q_p$ share the same metric 
    Each 
    \linkto{cosets_are_clopen}{$p^n \Z_p$ is clopen in $\Q_p$}.
    We first note that $\Q_p$ is locally compact at $0$
    since $\Z_p$ is an open compact neighbourhood of $0$.
    Furthermore, for any $a \in \Q_p$, 
    $a + \star$ is a homeomorphism so
    the coset $a + \Z_p$ 
    is the image of an open compact set which is open and compact.
    Clearly $a \in a + \Z_p$. 
    Hence $\Q_p$ is locally compact.

    Clearly $\bar{\io(\Q)} \subs \Q_p$
    Let $x \in \Q_p$,
    then $x = p^n u$ for $n \in \N $ and $u \in \Z_p$ a unit.
    Then 
    $\linkto{Z_dense_in_Z_p}{u \in \bar{\io(\Z)}} \subs \bar{\io(\Q)}$
    and so $x \in p^n \bar{\io(\Q)} \subs \bar{\io(\Q)}$.
    Thus $\Q$ is dense in $\Q_p$.

    $\Q_p$ is complete: take a Cauchy sequence in $\Q_p$.
    Let $\de = 1$, 
    then there exists $N \in \N$ such that for any $n, m \in \N$,
    if $N \le n \le m$ then $\abs{x_m - x_n} \le 1$.
    Hence the sequence $(x_m)_{N \le m} \subs x_N + \Z_p$ 
    which is compact as it is an image of the homeomorphism 
    $x_m + \star$.
    Hence there is a subsequence converging to a limit in 
    $x_m + \Z_p$, 
    and applying Cauchy we conclude this is the limit of the 
    original sequence.
\end{proof}

\begin{prop}[Series converge iff terms converge]
    Let $x : \N \to \Q_p$ be a sequence.
    Then $x$ converges if and only if 
    $\lim_{n \in \N}(x(n+1) - x(n)) = 0$.
\end{prop}
\begin{proof}
    Since $\Q_p$ is complete
    it suffices to show that $x$ is Cauchy if and only if 
    $\lim_{n \in \N}(x(n+1) - x(n)) = 0$.
    The forward implication is straightforward.
    For the other direction take $\de \in \R_{>0}$.
    By assumption 
    \[\exists N \in \N, \forall n \in \N_{>N}, 
    \abs{x(n+1) - x(n)} < \frac{\de}{2}\]
    Let $n, m \in \N$ be such that $N \le n \le m$.
    By induction we can show that 
    $\abs{x(m) - x(n)} \le \frac{\de}{2} < \de$,
    using $\abs{x + y} \le \max(\abs{x},\abs{y})$ 
    for the induction.
\end{proof}

\section{p-adic Equations}

\begin{prop}[Non-empty projective limits]
    \link{non_empty_projective_limits}
    Suppose $F : (\N,\le) \to \CC$ is a projective system.
    Denote $\fall{m}{n}$ as the image map in $\CC$ 
    from $F(n) \to F(m)$.
    Suppose that for every $n \in \N$ the object
    $F(n)$ in $\CC$ is finite and non-empty.
    Then the projective limit 
    \[\varprojlim F := \set{
        x \in \prod_{n\in \N} F(n) \st \forall n \in \N, 
        \fall{n+1}{n}x_{n+1} = x_n}\]
    is non-empty.
    Conversely if the projective limit is non-empty
    then each $F(n)$ is non-empty.
\end{prop}
\begin{proof}
    The trick is to construct a surjective projective system where
    the image objects are subsets of each $F(n)$.
    Let $n \in \N$.
    Suppose for a contradiction that 
    \[\forall k \in \N, \exists l \in \N_{\ge k}, 
    \fall{n}{n+l} D_{n+l} \ne \fall{n}{n+k} D_{n+k}\]
    Then by induction we can show that 
    \[\forall k \in \N, \exists l \in \N_{\ge k}, 
    \fall{n}{n+l} D_{n+l} \subset \fall{n}{n+k} D_{n+k}\]
    Since $D_n$ is finite and each 
    $\fall{n}{n+k} D_{n+k} \subs D_n$,
    we can conclude by induction that there exists 
    $k \in \N$ such that 
    $\fall{n}{n+k} D_{n+k} = \nothing$,
    which implies that $D_{n+k}$ is empty,
    a contradiction.
    Hence 
    \[\exists k \in \N, \forall l \in \N_{\ge k}, 
    \fall{n}{n+l} D_{n+l} = \fall{n}{n+k} D_{n+k}\]
    The sets `become constant'.
    We define a functor $G : (\N,\le) \to \CC$
    sending $n \mapsto \fall{n}{n+k} D_{n+k}$ and
    with the same image maps as $F$.
    This functor is well-defined and surjective because
    for any $n\in \N$, 
    using the `becomes constant' property of $G(n + 1)$
    we can show that $\fall{n}{n+1} G(n+1) = G(n)$.
    
    Let $x_0 \in G(0)$, 
    which is non-empty as it is the 
    image of a non-empty set $g(k)$ for some
    $k \in \N$.
    By induction we can find $x_n \in G(n)$ 
    for each $n \in \N$ such that 
    $\fall{n}{n + 1} x_{n+1} = x_n$.
    Hence $(x_n)_{n \in \N} \in \varprojlim G$.
    Since each $x_n \in F(n)$, 
    $(x_n)_{n \in \N} \in \varprojlim F$.

    The converse is immediate from the previous proposition.
\end{proof}

\begin{nttn}
    For $\phi: A \to B$ a ring morphism,
    $S$ a finite subset of $A[x_1, \dots, x_m]$,
    and
    \[f = \sum_{\la \in S} \la 
    \prod_{i = 1}^m (x_i)^{r_{i,\la}} 
    \quad \in A [x_1, \dots, x_m]\]
    we write $\phi (f)$ to mean 
    \[\sum_{\la \in S} \phi(\la) 
    \prod_{i = 1}^m (x_i)^{r_{i,\la}} 
    \quad \in B [x_1, \dots, x_m]\]
\end{nttn}

\begin{prop}[Vanishing commutes with limit]
    \link{vanishing_commutes_with_limit}
    Let $I \subs \Z_p [x_1, \dots, x_m]$.
    Then \[\V(I,\Z_p) \iso \varprojlim_{n \in \N} \V(\ep_n(I),A_n)\]
    via $(a_1,\dots, a_m) \in \V(I)$
    being sent to
    $(\ep_n(a_1),\dots, \ep_n(a_m))_{n\in\N} \in \varprojlim \V(\ep_n(I),A_n)$.

    In particular $\V(I)$ is non-empty
    if and only if for all $n \in \N$, 
    $V_n := \V (\ep_n(I))$ is non-empty,
    where $\ep_n(I)$ denotes $\set{\ep_n(f) \st f \in I}$.
\end{prop}
\begin{proof}
    Note that $(a_1, \dots, a_m) \in (\Z_p)^m$, 
    if and only if for all $i \in \set{1, \dots, m}$,
    $a_i \in \varprojlim A_n$
    if and only if 
    for all $i \in \set{1, \dots, m}, n \in \N$,
    $\ep_n(a_i) \in A_n$ and $\fall{n}{n+1}\ep_{n+1}(a_i) = \ep_n(a_i)$.
    This is if and only if for all $n \in \N$,
    \[(\ep_n(a_1), \dots, \ep_n(a_m)) \in A_n^m
    \text{and} \fall{n}{n+1}(\ep_{n+1}(a_1), 
    \dots, \ep_{n+1}(a_m)) = \ep_n(a_i)\]
    which is if and only if 
    $(\ep_n(a_1), \dots, \ep_n(a_m))_{n \in \N} \in \varprojlim (A_n^m)$.
    Hence we have an isomorphism of rings 
    \[(\Z_p)^m = (\varprojlim{A_n})^m \iso \varprojlim (A_n^m)\]

    We first show that the functor $V$ mapping $n \mapsto V_n$ 
    and $n \le m$ to $\fall{n}{m} : V_m \to V_n$
    is a projective system.
    We just need to show that 
    \[\forall n \in \N, \forall a \in V_{n+1}, 
    \fall{n}{n+1} a \in V_n\]
    Indeed if $a \in V_{n+1}$ then \footnote{
        For $\phi: A \to B we $ and $a \in A^m$ we write 
        $\phi(a) = \phi(a_1, \dots, a_m) = (\phi(a_1), \dots, \phi(a_m))$.
        In our projective system we use this notation for $\fall{n}{n+1}$.
    }
    \begin{align*}
        \ep_n (f) \circ \fall{n}{n+1}(a) 
        & = \fall{n}{n+1} \circ \ep_{n+1} 
        (f) (\fall{n}{n+1}(a))
        & \\
        &= \fall{n}{n+1} \brkt{\ep_{n+1} (f) (a)}
        & \text{ verify this}\\
        &= \fall{n}{n+1} (0) = 0 
        & \text{ since } a \in V_{n+1}
    \end{align*}
    Hence this forms a projective system with each $V_n$ 
    finite (since they are respectively subsets of $A_n$).
    
    Claim: $\varprojlim V$ is isomorphic to $\V(I)$
    via the isomorphism 
    \[(\Z_p)^m \iso \varprojlim (A_n^m)\]

    \begin{align*}
        (a_1, \dots, a_m) \in \V(I) \subs (\Z_p)^m 
        &\iff \forall f \in I, f(a_1, \dots, a_m) = 0 \in \Z_p\\
        &\iff \forall n \in \N, \forall f \in I,  
        \ep_n(f(a_1, \dots, a_m)) = 0 \in A_n\\
        &\iff \forall n \in \N, \forall f \in I, 
        \ep_n(f)(\ep_n(a_1), \dots, \ep_n(a_m)) = 0 \in A_n\\
        &\iff (\ep_n(a_1), \dots, \ep_n(a_m))_{n \in \N} \in  
        \varprojlim V
    \end{align*}
\end{proof}

\begin{dfn}
    For $R$ a ring $(a_1, \dots, a_m) \in R^m$ is primitive if
    there exists $i \in \set{1,\dots,m}$ such that $a_i$ is a unit.
    For the cases $R = A_n$ or $R = \Z_p$, 
    elements are non-primative if and only if for all 
    $i \in \set{1,\dots,m}$, $a_i \in p R^m$.
\end{dfn}

\begin{prop}
    Let $I \subs \Z_p[x_1,\dots, x_m]$ be such that 
    $\forall f \in I, f$ is homogeneous.
    Then the following are equivalent:
    \begin{enumerate}
        \item There exists a non-zero $a \in \V(I,\Q_p)$.
        \item There exists a primitive $a \in \V(I,\Z_p)$.
        \item For each $n \in \N$, 
            there exists a primitive $a \in \V(\ep_n(I),A_n)$.
    \end{enumerate}
\end{prop}
\begin{proof}
    2. implies 1. is straightforward. 
    If 1. is true then there exists a non-zero 
    $a = (a_1,\dots,a_m) \in (\Q_p)^m$
    such that for any $f \in I$, $f(a)=0$.
    Define $b = p^{-h} a$ where $h = \min_{1 \le i \le m} (v_p(a_i))$.
    This is well-defined as all $a_i$ are non-zero.
    $b$ is in $(\Z_p)^m$: for any $i \in \set{1, \dots ,m}$,
    $a_i = p^{v_p(a_i)} u_i$ for a unit $u_i \in \Z_p$ and so
    $b_i = p^(v_p - h) u_i$ with $0 \le v_p - h = v_p(b_i)$
    since $h$ was the minimum.
    $b$ is primitive: 
    there exists an $i$ that minimises $v_p(a_i)$.
    Then $b_i = p^{-h} a_i = p^{v_p(a_i) - h} u_i = u_i$ 
    is a unit in $\Z_p$.
    $b$ is in the vanishing $\V(I,\Z_p)$ because $f$ is homogeneous.
    (Write out $f$ as a sum and use the fact that the powers add to 
    the degree of $f$.)

    We show 2. if and only if 3. by considering the subsets
    $P(I, \Z_p)$ and $P(\ep_n(I),A_n)$, 
    the primitive elements of the vanishings.
    The $P(\ep_n(I),A_n)$ form a projective system with limit
    $\varprojlim P(\ep_n(I),A_n) \iso P(I, \Z_p)$ via the same
    isomorphism. 
    Then \linkto{non_empty_projective_limits}{
        $P(I, \Z_p)$ is non-empty if and only if for all $n \in \N$,
        $P(\ep_n(I),A_n)$ is non-empty.}
\end{proof}

\begin{prop}[Taylor's theorem]
    \link{taylor}
    If $R$ be a ring,
    $f\in R[x]$
    and $a \in \Z_p$,
    there exists a $g \in R[x]$
    such that 
    \[f(x) - f(a) = f'(a)(x - a) + g(x) (x - a)^2\]
\end{prop}
\begin{proof}
    Rephrase the statement as 
    \[f(x) - f(a) = f'(a)(x - a) \quad \mod (x - a)^2\]
    We show that for any $n$, $f = x^n$ satisfies the above.
    If $n = 0$ then we can pick $g(x) = 0$ and we are done.
    For the induction step we assume there exists 
    $g \in R[x]$ such that 
    \[x^n - a^n = n a^{n-1} (x-a) + g(x)(x - a)^2\]
    Suffices to show that 
    \[ \frac{x^{n+1} - a^{n+1}}{x-a}= (n+1) a^n \quad \mod (x - a)\]
    Then \begin{align*}
        \frac{x^{n+1} - a^{n+1}}{x-a}&= x^n + \dots + a^n\\
        &= \sum_{k=0}^n x^k a^{n-k} \quad \mod (x - a)^2\\
        &= \sum_{k=0}^n a^n \quad \mod (x - a)^2\\
        &= (n+1) a^n \quad \mod (x - a)^2
    \end{align*}
    Hence it is true for all monomials.
    Now let $f = \sum_n \la_n x^n$ be any polynomial.
    Then 
    \begin{align*}
        f(x) - f(a) = \sum_n \la_n (x^n - a^n)\quad \mod (x - a)^2\\
        &= \sum_n \la_n n a^{n-1} (x-a)\quad \mod (x - a)^2\\
        &= (x-a) \sum_n \la_n n a^{n-1}\quad \mod (x - a)^2\\
        &= (x-a)f'(a)\quad \mod (x - a)^2
    \end{align*}
\end{proof}

\begin{prop}[Newton's Method]
    \link{newton}
    Let $f \in \Z_p[x]$, $a \in \Z_p$
    conceptually:
    Suppose for $f'(a) \le 1$.
    Then there exists $y \in \Z_p$ such that
    \begin{enumerate}
        \item $\abs{f'(a)(y - a)} \le \abs{f(a)}$
        \item $\abs{f(y)} \le \frac{\abs{f(a)}}{p}$
        \item $\abs{f'(y)} = \abs{f'(a)}$
    \end{enumerate}
    Hence we have $y$ such that it is close to $a$,
    $f(y)$ is `much' closer to $0$, 
    and the derivative is the same size.

    Elementarily:
    Suppose $b,c \in \Z_p, n, k \in \Z$.
    Suppose $0 \le 2k < n$, $f(a) = p^n b$, $f'(a) = p^{k} c$ 
    and $c$ is a unit.
    Then there exists $y \in \Z_p$ such that 
    \[
        y - a \in p^{n-k} \Z_p \quad
        f(y) \in p^{n+1} \Z_p, \quad v_p(f'(y)) = k, 
    \]
\end{prop}
\begin{proof}
    Take $y = a - p^{n-k} c^{-1} b$.
    Clealy $y - a \in p^{n-k} \Z_p $.
    By \linkto{taylor}{Taylor's formula}
    \begin{align*}
        f(y) - f(a) &=  
        - f'(a) p^{n-k} c^{-1} b + g(y) c^{-2} b^2 (p^{n-k})^2\\
        \implies f(y) - p^n b &= - p^{k} b p^{n-k} z + 
        g(y) c^{-2} b^2 p^{2n - 2k}\\
        \implies f(y) &= c^{-2} b^2 g(y) p^{2n - 2k}
    \end{align*}
    Hence $f(y) \in p^{2n + 1} \Z_p$ if and only if
    $2n + 1 \le 2n - 2k$ if and only if $2k + 1 \le n$,
    which is true.

    To check that  $v_p(f'(y)) = k$ we use Taylor's formula again:
    \[f'(y) - f'(a) = f''(a)(y - a) + g(y) (y - a)^2\]
    Hence 
    \begin{align*}
        f'(y) &= 
        f'(a) - f''(a) p^{n-k} c^{-1} b + g(x)p^{2n-2k}c^{-2}b^2\\
        &= p^k c - (f''(a) c^{-1} b + g(x)p^{n-k}c^{-2}b^2)p^{n-k}\\
        &= p^k (c - \star p^{n-2k})
    \end{align*}
    where $\star \in \Z_p$.
    Hence $c - \star p^{n-2k}$ is a unit since $p$ does not divide it.
    Thus $v_p(f'(y)) = k$. 
\end{proof}

\begin{prop}[Polynomials are continuous]
    \link{polynomials_continuous}
    The maps $\star + \star : (\Q_p)^2 \to \Q_p$ and 
    $\star \cdot \star : (\Q_p)^2 \to \Q_p$ are continuous.
    Hence by induction polynomials are continuous maps.
\end{prop}
\begin{proof}
    Standard.
    For product use the trick
    \[ab - cd = a(b - d) + b(a - c) +(a - c)(d - b)\]
\end{proof}

\begin{prop}[General Hensel]
    \link{gen_hensel}
    If $f \in \Z_p[x_1,\dots, x_m]$ and there exist 
    $a \in (\Z_p)^m$,
    $n, k \in \Z$ such that $0 \le 2k < n$ and 
    $f(a) \in p^n \Z_p$
    and there exists $j \in \set{1, \dots, m}$ such that 
    $v_p (\dbd{f}{x_j}(a)) = k$,
    then there exists $y \in (\Z_p)^m$ such that
    \[a - y \in p^{n-k} \Z_p \quad \text{ and } \quad f(y) = 0\]
\end{prop}
\begin{proof}
    Case $m = 1$ and let $f \in \Z_p[x_1]$, 
    $a \in \Z_p, n, k \in \Z$ such that $f(a) \in p^n \Z_p$
    such that 
    $v_p (\dbd{f}{x_1}(a)) = v_p(f'(a))= k$.
    Let $y_n = a$.
    By induction with \linkto{newton}{Newton's Method} 
    at each step, 
    we obtain for each $l \in \N_{> n}$ a 
    $y_l \in \Z_p$ such that $f(y_l) \in p^{m} \Z$,
    $v_p(f'(y_m)) = k$ and 
    $y_{l} - y_{l-1} \in p^{l-1-k} \Z_p$.
    The $(y_m)_{m \in \N}$ is a sequence in $\Z_p$
    which is Cauchy since each $y_{l} - y_{l-1} \in p^{l-1-k} \Z_p$
    so $\abs{y_{l} - y_{l-1}} \le p^{k + 1 - l} \to 0$ 
    as $l \to \infty$.
    Since $\Z_p$ is complete this converges to $y \in \Z_p$.
    It is clear that $\abs{y - y{l}} \le p^{k-l}$
    for each $l$. 
    In particular $\abs{y - a} \le p^{k-n}$
    hence $a - y \in p^{n-k}$.
    Furthermore since 
    \linkto{polynomials_continuous}{$f$ is continuous}
    and $f(y_n)$ are in shrinking balls around $0$,
    \[f(y) = f(\lim_{n \to \infty} y_n) = 
    \lim_{n \to \infty} f(y_n) = 0\]
    
    For the case $1 < m$ 
    we reduce it to the same situation as above.
    Suppose $f \in \Z_p[x_1,\dots, x_m]$,
    $a \in (\Z_p)^m$,
    $n, k \in \Z$ such that $f(a) \in p^n \Z_p$
    and there exists $j \in \set{1, \dots, m}$ such that 
    $v_p (\dbd{f}{x_j}(a)) = k$.
    Then take 
    $f(a_1,\dots, a_{j-1}, x_j, a_{j+1},\dots, a_m) \in \Z_p[x_j]$,
    $f$ with its variables substituted for $a_i$ except for when $i = j$.
    This satisfies the conditions of the first part
    so we are done.
\end{proof}

\begin{cor}[Hensel]
    \link{hensel}
    Let $f \in \Z_p [x_1,\dots,x_m]$, 
    suppose there exists $c \in (p \Z_p)^m$
    such that $\ep_1(f(c)) = 0$ and 
    there exists a $j \in \set{1,\dots m}$
    such that $\dbd{f}{x_j}(c) \ne 0$
    then there exists a $c^* \in (\Z_p)^m$ 
    such that $f(c^*) = 0$ and 
    $\ep_1(c^* - c) = 0$.
\end{cor}
\begin{proof}
    Apply \linkto{gen_hensel}{general Hensel with}
    $n = 1$ and $k = 0$.
\end{proof}

\begin{cor}[Quadratic forms for $p \ne 2$]
    Suppose $p \ne 2$, 
    $A \in (p\Z_p)^{n \times n}$
    such that for all 
    $i,j \in \set{1,\dots,m}, A_{ij} = A_{ji}$
    and $\det A$ a unit. 
    Let
    \[
        f = x^T A x = 
        \sum_{i = 1}^m \sum_{j = 1}^m A_{ij} x_i x_j
        \in \Z_p[x_1,\dots,x_m]
    \]
    If for any $a \in \Z_p$,
    there exists primitive $c \in (\Z_p)^m$
    such that $ep_1(f(c) - a) = 0$
    then there exists $c^* \in (\Z_p)^m$
    such that $f(c^*) = 0$ and $\ep_1(c^* - c) = 0$.   
\end{cor}
\begin{proof}
    By \linkto{hensel}{Hensel} applied to $g(x) := f(x) - a$
    it suffices to show that
    there exists a $j \in \set{1,\dots, m}$ such that 
    $\dbd{f}{x_j}(c) \ne 0$.
    Suppose not.
    Then for any $j \in \set{1,\dots, m}$
    \[0 = \dbd{f}{x_j}(c) = 2 \sum_{i \in S} A_ij c_i\]
    Since $p \ne 2$ we have that for all $j$
    \[0 = \sum_{i \in S} \ep_1(A_ij) \ep_1(c_i) = 
    \ep_1 (A_j) \ep_1(c)\]
    Hence \[0 = \ep_1 (A) \ep_1 (c)\]
    Since the determinant of $A$ is a unit,
    the determinant of $\ep_1 (A)$ is a unit
    (determinant commutes with ring morphisms).
    Thus multiplying by the adjugate of $\ep_1 (A)$ 
    we obtain $0 = \ep_1(c)$.
    This is a contradiction as $c$ is primitive.
\end{proof}

\begin{cor}[Quadratic forms for $\Z_2$]
    Suppose $A \in (2 \Z_2)^{n \times n}$
    such that for all 
    $i,j \in \set{1,\dots,m}, A_{ij} = A_{ji}$. 
    Let
    \[
        f = x^T A x = 
        \sum_{i = 1}^m \sum_{j = 1}^m A_{ij} x_i x_j
        \in \Z_2[x_1,\dots,x_m]
    \]
    If for any $a \in \Z_2$,
    there exists primitive $c \in (\Z_2)^m$
    such that $ep_3(f(c) - a) = 0$ and
    \[\det(A) \text{ is a unit of } \Z_2 \quad \OR \quad 
    \exists j \in \set{1,\dots,m}, \ep_2(\dbd{f}{x_j}(c)) \ne 0\]
    then there exists $c^* \in (\Z_p)^m$
    such that $f(c^*) = 0$ and $\ep_1(c^* - c) = 0$.   
\end{cor}
\begin{proof}
    We show that 
    \[\det A \text{ is a unit of } \Z_2 \quad \implies \quad 
    \exists j \in \set{1,\dots,m}, \ep_2(\dbd{f}{x_j}(c)) \ne 0\]
    Suppose not.
    \begin{align*}
        &\quad \forall j \in \set{1,\dots,m}, \ep_2(\dbd{f}{x_j}(c)) = 0\\
        &\implies \forall j, \ep_2(2 \sum A_{ij} c_i) = 0\\
        &\implies \forall j, 2 \ep_2(A_j)\ep_2(c) = 0\\
        &\implies 2 \ep_2(A)\ep_2(c) = 0\\
        &\implies 2 \ep_2(c) = 0 \quad \text{since } 
        \det A \text{is a unit}\\
        &\implies \ep_2(c) = 0 \OR \ep_2(c) = 2
        &\implies \ep_1(c) = 0 \quad \text{ a contradiction}
    \end{align*}
    Hence in either case $\exists j, \ep_2(\dbd{f}{x_j}(c)) \ne 0$
    We can then apply \linkto{gen_hensel}{general Hensel} 
    with $n = 3, k < 2, g(x) = f(x) - a$ and obtain $c^*$.
\end{proof}




\end{document}