\section{Quadratic reciprocity}
\begin{prop}[Exact sequence]
    If $K$ is a finite field,
    \begin{itemize}
        \item If $\Char{K} = 2$ then all elements are square.
        \item If $\Char{K} \ne 2$ 
        then the non-zero squares form a subgroup of index $2$,
        and is the kernel of the group morphism 
        $x \to x^{\frac{q-1}{2}}$ into $\<-1\>$.
    \end{itemize}
    I can't be bothered to make the exact sequence.%
\end{prop}
\begin{proof}~
    \begin{itemize}
        \item If $\Char{K} = 2$ then the 
        \linkto{frobenius}{Frobenius map} $\si_2 : x \mapsto x^2$
        is an automorphism of $K$. 
        Hence the preimage of any element squares to that element.
        \item If $\Char{K} \ne 2$ then generate $K^* = \<g\>$ 
        since it is cyclic.
        The map $x \to x^{\frac{q-1}{2}}$ 
        has kernel $\set{x \in K \st x \text{ square}}$ since
        (writing any element as a multiple of $g$)
        \[
            g^n \in \ker \iff g^\frac{n(q-1)}{2} = 1 \iff 
            q - 1 \mid \frac{n(q - 1)}{2} \iff n \text{ even} 
            \iff x \text{ square}
        \]
        We check where the generator $g$ is sent. 
        If $g^{\frac{q - 1}{2}} = 1$ then the order of $g$ 
        is less than $q - 1$ which is a contradiction
        hence the image is non-trivial.
        Any element of the image of the map squares to $1$
        hence solves $x^2 - 1 = 0$,
        which only has two solutions in $K$.
        Thus the image is $\<-1\>$ and the index of the kernel is $2$.
    \end{itemize}
\end{proof}

\begin{dfn}[Legendre symbol]
    If $p$ is prime that is not $2$ and $x \in \F_p$ then
    \[\legen{x}{p} := 
    \begin{cases}
        x^{\frac{p-1}{2}} &, x \text{ unit}\\
        0 &, x = 0
    \end{cases}\]
    Check that for each $p$ this is a homomorphism $\F_p \to \<-1\>$.
\end{dfn}

\begin{dfn}[$\ep(n)$]
    If $n \in \Z$ is odd 
    \[\ep(n):=\frac{n-1}{2} (\mathrm{mod} 2)\]
\end{dfn}

\begin{prop}[Computations]
    \begin{align*}
        \legen{1}{p} =& 1\\
        \legen{-1}{p} =& (-1)^{\ep(p)}
    \end{align*}
\end{prop}

\begin{prop}[Quadratic reciprocity]
    Let $l \ne p$ be primes that aren't $2$.
    Then \[\legen{l}{p}\legen{p}{l} = (-1)^{\ep(l)\ep(p)}\]
\end{prop}
\begin{proof}
    Let $w$ be order $l$ element of $\Om$,
    the algebraic closure of $\F_p$.
    For $x \in \F_l$ write $w^x$ to be $w^r$ for any $r \in \Z$ such that
    $x = \bar{r} \in \F_l$ (independant of choice of $r$ by $w^l = 1$).
    Let 
    \[y = \sum_{x\in \F_l} \legen{x}{l} w^x \in \Om\]

    We first show that $y^2 = (-1)^{\ep(l)} \bar{l}$, 
    where $\bar{l} \in \F_p$.
    \begin{align*}
        y^2 =& 
        (\sum_{x \in \F_l} \legen{x}{l} w^x)
        (\sum_{y \in \F_l} \legen{y}{l} w^y)\\
        =& \sum_{x \in \F_l} \sum_{y \in \F_l} 
        \legen{x}{l} w^x \legen{y}{l} w^y\\
        =& \sum_{x \in \F_l} \sum_{y \in \F_l} \legen{xy}{l} w^{x+y}\\
        =& \sum_{u \in \F_l} \sum_{x \in \F_l} \legen{x(u - x)}{l} w^u
    \end{align*}
    Case on what $x$ is:
    \begin{align*}
        x \ne 0 \implies& \legen{x(u - x)}{l} =& \legen{xu - x^2}{l}\\
            & =& \legen{x^2}{l} \legen{-1}{l} \legen{1-\frac{u}{x}}{l}\\
            & =& x^{p - 1} \legen{-1}{l} \legen{1 - \frac{u}{x}}{l}\\
            & =& (-1)^{\ep(l)} \legen{1 - \frac{u}{x}}{l}
    \end{align*}
    If $x = 0$ then clearly $\legen{x(u - x)}{l} = 0$.
    Hence 
    \[
        y^2 = \sum_{u \in \F_l} \sum_{x \in \F_l^*} 
        (-1)^{\ep(l)} \legen{1 - \frac{u}{x}}{l}
        = (-1)^{\ep(l)} \sum_{u \in \F_l}  \sum_{x \in \F_l^*}
        \legen{1 - \frac{u}{x}}{l}
    \]
    Given $x \ne 0$, case on what $u$ is:
    \begin{align*}
        u = 0 \implies &\sum_{x \in \F_l^*} \legen{1 - \frac{u}{x}}{l} \\
        = &\sum_{x \in \F_l^*} \legen{1}{l} \\
        = &\sum_{x \in \F_l^*} 1\\
        = & \bar{l} -1
    \end{align*}
    \begin{align*}
        u \ne 0 \implies &\sum_{x \in \F_l^*} \legen{1 - \frac{u}{x}}{l}\\
            = &\sum_{x \in F_l^*} \legen{1 - \frac{1}{x}}{l}\\
            = &\sum_{s \in \F_l^*} \legen{1 - s}{l}\\
            = &\sum_{s \in \F_l \setminus \set{1}} \legen{s}{l}\\
            = &\sum_{s \in \F_l} \legen{s}{l} - \legen{1}{l}\\
            = & - 1 
    \end{align*}
    Since the index of the kernel of $\legen{\star}{l}$ is $2$,
    and the cosets have equal cardinality.
    Hence 
    \begin{align*}
        y^2 (-1)^{\ep(l)} &= \sum_{u \in \F_l}  \sum_{x \in \F_l^*}
            \legen{1 - \frac{u}{x}}{l}\\
            &= \bar{l} - 1 - \sum_{u \in \F_l^*} w^u\\
            &= \bar{l} - (1 + w + w^2 + \dots + w^l)
    \end{align*}
    since $l$ is prime. 
    Note that $0 = w^l - 1 = (w+1)(1+w + \dots + w^l)$.
    Hence $1+w + \dots + w^l = 0$ and $y^2 = (-1)^{\ep(l)} \bar{l}$.

    Next we show that $y^{p - 1} = \legen{p^-1}{l}$.
    \begin{align*}
        y^p = & \sum_{x \in \F_l} {\legen{x}{l}}^p w^xp & \text{ `Freshman's dream'}\\
            = & \sum_{x \in \F_l} {\legen{x}{l}} w^xp & 
            \legen{x}{l}= \pm 1 \text{ and } p \text{ is odd}\\
            = & \sum_{z \in \F_l} \legen{z p^{-1}}{l} w^z\\
            = & \legen{p^{-1}}{l}(\sum_{z \in \F_l} \legen{z}{l} w^z)\\
            = & \legen{p^{-1}}{l} y
    \end{align*}
    Hence \[y^{p - 1} = \legen{p^{-1}}{l} = (\legen{p{l}})^{-1}\]
    thus 
    \begin{align*}
        \legen{l}{p}\legen{p}{l} =& \legen{l}{p} y^{1 - p}\\
        =& \legen{l}{p} (y^2)^{\frac{1 - p}{2}}\\
        =& \legen{l}{p} ((-1)^{\ep(l)} \bar{l})^{\frac{1 - p}{2}}\\
        =& \legen{l}{p} (\legen{(-1)^{\ep(l)}l}{p})^{-1}\\
        =& (\legen{(-1)^{\ep(l)}}{p})^{-1}\\
        =& (-1)^{\ep(l)\ep(p)}
    \end{align*}
\end{proof}




%%%%%%%%%%%%%%%%%%%%%%%%
\chapter{$p$-adic Fields}
\section{$p$-adic Integers and Rationals}
\begin{dfn}[Projective system $A$]
    Define a contravariant functor $A : (\N,\le) \to \RING$ 
    such that for each $n$
    \[A_n := \Z / p^n \Z \quad \text{ and } 
    \quad \pi_n : \Z \to A_n \text{ is the projection}\]
    and for any $n$ such that $1 \le n$, 
    there exists a surjective ring morphism 
    $\phi_n : A_n \to A_{n-1}$ such that
    $\phi_n \circ \pi_n = \pi_{n-1}$ and 
    $\ker(\phi_n) = p^{n-1} A_n$.
\end{dfn}
\begin{ex}
    Check that such a $\phi_n$ exists.
\end{ex}

\begin{dfn}[$p$-adic integers]
    Let 
    \[\Z_p = \{x \in \prod_{n \in \N} A_n \st 
    (\forall n \in \N, x_n \in A_n) \AND
    (\forall n > 0, \phi_n(x_n) = x_{n-1})\}\]
    Define addition and multiplication pointwise. 
    Verify that this $\Z_p$ is a ring with $0 = (0)_{n \in \N}$
    and $1 = (1)_{n \in \N}$.

    For each $n \in \N$ let $\ep_n : \Z_p \to A_n$ be the 
    ring morphisms mapping $x \mapsto x_n$.
    Note that by definition $\phi_n \circ \ep_n = \ep_{n-1}$.

    In addition,
    provide each $A_n$  with the discrete toplogy,
    giving $\prod_{n \in \N} A_n$ the product topology
    and $\Z_n$ the subset topology.
\end{dfn}

\begin{prop}[$\Z_p$ is compact]
    Since each $A_n$ is finite, 
    each $A_n$ is compact.
    Hence by Tychonoff's theorem
    the product is compact.
    Since closed in compact is compact
    we just need to show that $\Z_p$ is closed.

    We want to write $\Z_p$ as the intersection of 
    closed sets 
    \[D_k:= \set{x \in \prod_{n\in\N} A_n \st \phi_k (x_k) = x_{k-1}}\]
    for $k \in \N$.
    Clearly 
    \[\bigcap_{k\in\N} D_k = \Z_p\]
    and 
    \[D_k = \bigcup_{x_{k-1} \in A_{k-1}} \brkt{\ep_{k-1}^{-1}(x_{k-1}) 
    \cap \bigcup 
    \set{\ep_k^{-1}(x_k) \st x_k \in A_k \AND \phi_k(x_k) = x_{k-1}}}\]
    Since each $\set{x_k}$ is closed in $A_k$, 
    each preimage $\ep_k^{-1}(x_k)$ is closed.
    Thus the finite union of the preimages
    \[\bigcup 
    \set{\ep_k^{-1}(x_k) \st x_k \in A_k \AND \phi_k(x_k) = x_{k-1}}\]
    is closed.
    Since each $\set{x_{k-1}}$ is closed in $A_{k-1}$, 
    each preimage $\ep_{k-1}^{-1}(x_{k-1})$ is closed.
    Thus intersection 
    \[\brkt{\ep_{k-1}^{-1}(x_{k-1}) 
    \cap \bigcup 
    \set{\ep_k^{-1}(x_k) \st x_k \in A_k \AND \phi_k(x_k) = x_{k-1}}}\]
    is closed.
    Hence the finite union is closed and $D_k$ is closed.
    Arbitrary intersection of closed is closed so $\Z_p$ 
    is closed and thus compact.
\end{prop}

\begin{prop}[Universal property of $\Z_p$]
    Suppose $R$ is a ring with ring morphisms 
    $\rho_n : R \to A_n$ for each $n \in \N$
    such that for each $n > 0$, 
    $\phi_n \circ \rho_n = \rho_{n-1}$.
    Then there exists a unique ring morphism $f : R \to \Z_p$
    such that for each $n$, 
    $\ep_n \circ f = \rho_n$.
\end{prop}
\begin{proof}
    If there exists such a map then it is unique:
    suppose $f,g$ both satisfy the given properties.
    Then for any $n$ and any $a \in R$
    $\ep_n \circ f (a)= \rho_n (a) = \ep_n \circ g (a)$.
    Thus $f(a) = g(a)$, by the property of products
    (if they agree on all the projections they are equal).

    For existance we let $a \in R$ and consider the set
    \[\bigcap_{n \in \N} {\ep_n^{-1} \circ \rho_n (a)}\]
    show that it has cardinality $1$,
    and let $f$ map $a$ to this unique element.
    If $x,y \in \bigcap_{n \in \N} {\ep_n^{-1} \circ \rho_n (a)}$
    then for any $n \in \N$, 
    $\ep_n (x) = \rho_n (a) = \ep_n (y)$.
    Thus $a = b$ by the property of products.
    Hence the cardinality is $\le 1$.

    To show that the set is non-empty,
    take $x = (\rho_n (a))_{n \in \N}$.
    This is in $\Z_p$ since for each $n>0$,
    $\phi_n \circ \rho_n(a) = \rho_{n - 1}(a)$.
    Also it is in the intersection since for each $n$,
    $\ep_n(x) = \rho_n(a)$. 
    Hence the cardinality is $1$.
    Hence $f$ is well-defined and for all $n \in \N$, 
    $\ep_n \circ f = \rho_n$.

    For any $n$, 
    \[\ep_n \circ f (a + b) = \rho_n (a + b) = \rho_n (a) + \rho_n (b)
    = \ep_n \circ f(a) + \ep_n \circ f(b) = \ep_n (f(a) + f(b))\]
    Hence by property of products $f(a + b) = f(a) + f(b)$
    and similarly for multiplication.
    Note that for any $n$, 
    $\ep_n \circ f(1) = \rho_n (1) = 1$.
    Hence $f(1) = 1$.
    Thus $f$ is a ring morphism.
\end{proof}

\begin{cor}[$\Z$ injects into $\Z_p$]
    Then there exists a unique injective ring morphism 
    $\io : \Z \to \Z_p$
    such that for each $n$, 
    $\ep_n \circ \io = \pi_n$.
\end{cor}
\begin{proof}
    By the previous theorem the morphism exists and is unique.
    It must send $1 \mapsto 1$ hence $\io(x) = 0$ would imply
    $\pi_n (x) = \ep_n \circ \io(x) = 0$ for all $n \in \N$.
    Hence for any $n \in \N$, 
    $p^n \mid x$.
    Thus $x = 0$.
\end{proof}

